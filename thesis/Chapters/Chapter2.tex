\chapter{Research} % Main chapter title

\label{Chapter2} 


\section{Static site generators}

Static site generators take your app and build in before serving to users. 
This means users receive plain HTML files. This moves a large computation burden from run time to build time which results in significantly faster load times for users.
Furthermore, this approach allows for more aggressive and efficient caching.

\subsection{Nextjs}

Nextjs\footnote{https://nextjs.org/} is a React framework. Not explicitly a static site generator but has support for it

\subsection{Gatsby}

Gatsby\footnote{https://www.gatsbyjs.com/} is a static site generator at heart. It might be harder to do runtime stuff with Gatsby

\subsection{Umi.js}

%----------------------------------------------------------------------------------------

\section{Deployment configuration}

\subsection{Store inside CMS}

This is a low-effort solution. The idea is to create a deployment schema config with well-defined properties. We use this schema as a content model inside the CMS.

Positive:

\begin{itemize}
	\item Non-development teams get a nice interface to change things while still using a controlled schema
	\item No need to create a custom frontend for this
	\item Very flexible, easy to add or change properties
\end{itemize}

Negative:

\begin{itemize}
	\item Very flexible, easy to create broken configurations if we do not properly validate.
	\item The CMS becomes an even more important part of the technology stack. I consider this a negative because currently Stampix uses a proprietary, closed source service.
\end{itemize}


\subsection{Storybook}

Stampix is using Storybook\footnote{https://storybook.js.org/} to create a component library for the mobile application. With Storybook, the operations team can change some parameters and then export a configuration.
This is nice because Storybook provides a sort of sandbox to play around and design a deployment, completely separated from any cloud resources.

Problem with this is that Storybook focuses mostly on styling. We need more than just styling.

%----------------------------------------------------------------------------------------

\section{Automated testing}

We create lots of different webapps, a different one for each client. Currently, we have no automated tests. This should change

Automated testing gives developers confidence their changes did not break anything. It can spot bugs before the code is even released

\subsection{Selenium}

Selenium\footnote{https://www.selenium.dev/} is an established project

\subsection{Cypress}

Cypress\footnote{https://www.cypress.io/} is pretty new and 'cool' 

\subsection{Protractor}

\subsection{Playwright}

Playwright\footnote{https://playwright.dev/} is

%----------------------------------------------------------------------------------------

\section{Deployment}

\subsection{Ansible}
\subsection{Terraform}
\subsection{Serverless}
\subsection{AWS Cloud Development Kit}
