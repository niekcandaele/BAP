\chapter{Intro} % Main chapter title

\label{Chapter1} % For referencing the chapter elsewhere, use \ref{Chapter1} 

%----------------------------------------------------------------------------------------

% Define some commands to keep the formatting separated from the content 
\newcommand{\keyword}[1]{\textbf{#1}}
\newcommand{\tabhead}[1]{\textbf{#1}}
\newcommand{\code}[1]{\texttt{#1}}
\newcommand{\file}[1]{\texttt{\bfseries#1}}
\newcommand{\option}[1]{\texttt{\itshape#1}}

%----------------------------------------------------------------------------------------

\section{How it used to be}

Stampix is a startup that prints photos. Stampix' customers are companies, these companies buy printcodes which they can then distribute to their users in context of marketing or loyalty campaigns. Every client gets their own branded web application.

This involves:

\begin{itemize}
	\item Storing all brand-related content in a CMS (Contentful)
	\item Pulling in all that content during app runtime
	\item Deployments for new clients require a lot of manual configuration / dev work
	\item There's no automated tests, which can cause broken deployments if not careful
\end{itemize}

This had a few problems which I will explain in detail later \ldots

%----------------------------------------------------------------------------------------

\section{Solutions}

Following are the methods used to improve this workflow. Each method will probably get it's own detailed chapter later?

\begin{itemize}
	\item Using a static site generator to build web app and assets during build time
	\item Automated testing (Selenium-like / snapshots / unit)
	\item Deploying each built application to AWS
\end{itemize}

\section{Requirements}

\subsection{Infrastructure as code}

A big pain point right now is that it takes a lot of manual (development) work to create new deployments.
We can solve this by automating the process, however it's not as simple as just building the frontend assets and uploading them to the cloud.

The operations and sales teams must be able to create deployments on their own and they must be able to control certain aspects of the final product.


\subsection{Asset managment}

Brand images, texts, ...

This is currently kept in a CMS. CMS' are made for this, which automatically gives us a lot of functionality. 

An additional pain point here is that the structure of the data is still manually managed. Stampix has asked to see if there is some sort of IaC solution possible for this.

\subsection{Deployment configuration}

Domain name (company.stampix.com), length of the campaign, total amount of prints bought, \dots

This should be a semi-structured document which can be easily edited by non-development teams which will then get sent to our IaC solution.

\subsection{Deployment stages}

There should be a clear distinction between deployments in production, staging or test. 
A common occurence right now is that the sales team will create demo's to use during their pitches. This has the risk that they change some config which breaks production apps. These demo's also get made, pitched and then prompty forgotten, never to be used again.

\begin{itemize}
	\item Production: deployments that are live, in actual use.
	\item Staging: deployments before going to production. Final checks happen here
	\item Test: sales demos, development builds, \dots anything else
\end{itemize}

\subsection{Modularity}

The web app must support 'plugins'. These plugins can literally be anything. They can include extra logic in the backend, extra pages in the frontend or a combination of both.

TODO: How will this be configured? Could be a multi-select in the deployment configuration

\subsubsection{NSFW check}

This plugin checks every order for photos that has \textbf{N}ot \textbf{S}afe \textbf{F}or \textbf{W}ork content. If an order includes content like this, it gets rejected.

\subsubsection{Thank you module}

After creating an order, the user is presented with a small form that asks if they would like to say thank you. The user can select an emoji to reflect their feelings. 
This module is particularly useful for creating metrics at the end of a campaign

