\chapter{Research} % Main chapter title

\label{Chapter2} 


\section{Static site generators}

Static site generators take your app and build in before serving to users. 
This means users receive plain HTML files. This moves a large computation burden from run time to build time which results in significantly faster load times for users.
Furthermore, this approach allows for more aggressive and efficient caching.

\subsection{Nextjs}

Nextjs\footnote{https://nextjs.org/} is a React framework. Not explicitly a static site generator but has support for it

\subsection{Gatsby}

Gatsby\footnote{https://www.gatsbyjs.com/} is a static site generator at heart. It might be harder to do runtime stuff with Gatsby

\subsection{Umi.js}

%----------------------------------------------------------------------------------------

\section{Automated testing}

We create lots of different webapps, a different one for each client. Currently, we have no automated tests. This should change

Automated testing gives developers confidence their changes did not break anything. It can spot bugs before the code is even released

\subsection{Selenium}

Selenium\footnote{https://www.selenium.dev/} is an established project

\subsection{Cypress}

Cypress\footnote{https://www.cypress.io/} is pretty new and 'cool' 

\subsection{Protractor}

\subsection{Playwright}

Playwright\footnote{https://playwright.dev/} is

%----------------------------------------------------------------------------------------

\section{Deployment}

\subsection{Ansible}
\subsection{Terraform}
\subsection{Serverless}
\subsection{AWS Cloud Development Kit}
